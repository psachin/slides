% Created 2012-12-22 Sat 22:11
\documentclass[bigger, presentation]{beamer}
\usepackage[utf8]{inputenc}
\usepackage[T1]{fontenc}
\usepackage{fixltx2e}
\usepackage{graphicx}
\usepackage{longtable}
\usepackage{float}
\usepackage{wrapfig}
\usepackage{soul}
\usepackage{textcomp}
\usepackage{marvosym}
\usepackage{wasysym}
\usepackage{latexsym}
\usepackage{amssymb}
\usepackage{hyperref}
\tolerance=1000
\usetheme{Frankfurt}   
\usecolortheme[RGB={0,104,139}]{structure}%deepskyblue
\usefonttheme{serif}  % or try serif, structurebold, ...
\setbeamertemplate{navigation symbols}[horizontal]
\setbeamertemplate{caption}[numbered]
\useinnertheme{rounded}
\setbeamercovered{transparent}
\usepackage{pgfpages}
\pgfpagesuselayout{resize to}[physical paper width=8in, physical paper height=6in]
\logo{\includegraphics[height=1cm,width=1cm]{scipyshiny_small.png}}
\usepackage{array}
\usepackage{graphicx}
\usepackage{hyperref}
\usepackage[english]{babel}
\usepackage{pxfonts}
\usepackage{listings}
\lstset{numbers=left,numbersep=6pt,numberstyle=\tiny,showstringspaces=false,aboveskip=-50pt,frame=leftline,keywordstyle=\color{black},commentstyle=\color{orange},stringstyle=\color{black},}
\date{today}
\subtitle{writing beamer presentation in org-mode}
\institute{Indian Institute of Technology, Bombay}
\providecommand{\alert}[1]{\textbf{#1}}

\title{presentation.org}
\author{sachin}
\date{\today}
\hypersetup{
  pdfkeywords={org mode, emacs, latex, beamer, pdf},
  pdfsubject={my first presentation made in org mode},
  pdfcreator={Emacs Org-mode version 7.8.02}}

\begin{document}

\maketitle

\section{Introduction}
\label{sec-1}
\begin{frame}
\frametitle{A simple slide}
\label{sec-1-1}

This slide consist of some text with a number of bullets points

\begin{itemize}
\item first the very important
\item some more text to go here. THis is really a crap. more goes
  here. and this is really a BUllshit
\end{itemize}

and the slide ends here
\end{frame}
\section{summary}
\label{sec-2}

  
\begin{frame}
\frametitle{Second slide}
\label{sec-2-1}

   texts in various formats

\begin{itemize}
\item this is in \textbf{bold}
\item this is \underline{underline}
\item this is \emph{emphasis}
\item and this is the \hyperref[www.google.ru]{link}
\end{itemize}
\end{frame}
\begin{frame}
\frametitle{quote}
\label{sec-2-2}

\begin{quote}
Emacs org-mode is a 
great presentation tool 
\begin{itemize}
\item Fast to beautiful slides
\item Arne Babenhauserheide
\end{itemize}

\end{quote}

     
\end{frame}
\section{tables}
\label{sec-3}
\begin{frame}
\frametitle{simple table}
\label{sec-3-1}



\begin{table}[htb]
\caption{A long table} \label{tab:long}
\begin{center}
\begin{tabular}{lrr}
\hline
 Name   &  Phone  &  Age  \\
\hline
 Peter  &   1234  &   17  \\
 Anna   &   4321  &   25  \\
\hline
 sds    &     ds  &   sd  \\
\hline
\end{tabular}
\end{center}
\end{table}



\begin{table}[htb]
\caption{A second table} 
\begin{center}
\begin{tabular}{c|c|c}
\hline
 Name   &  Phone  &  Age  \\
\hline
 Peter  &   1234  &   17  \\
 Anna   &   4321  &   25  \\
\hline
 sds    &     ds  &   sd  \\
\hline
\end{tabular}
\end{center}
\end{table}
\end{frame}
\section{images}
\label{sec-4}
\begin{frame}
\frametitle{simple image}
\label{sec-4-1}



  \begin{figure}[htb]
  \centering
  \includegraphics[width=5cm,angle=0]{/home/sachin/github/slides/org/scipyshiny_small.png}
  \caption{\label{fig:SED-HR4049}Scipy logo}
  \end{figure}
\end{frame}
\begin{frame}
\frametitle{second image (resized \& rotated)}
\label{sec-4-2}


  \begin{figure}[htb]
  \centering
  \includegraphics[width=2cm,angle=50]{/home/sachin/github/slides/org/scipyshiny_small.png}
  \caption{\label{fig:SED-HR4049}Scipy logo}
  \end{figure}
\end{frame}
\section{blocks}
\label{sec-5}
\begin{frame}
\frametitle{I love Python}
\label{sec-5-1}
\begin{block}{Why?}
\label{sec-5-1-1}

\begin{itemize}
\item Lets me focus on the Problem
\item Interactive
\item Readable
\end{itemize}
\end{block}
\end{frame}
\begin{frame}
\frametitle{A more complex slide}
\label{sec-5-2}

   This slide illustrates the use of Beamer blocks.  The following text,
   with its own headline, is displayed in a \textbf{block}
\begin{block}{Org mode increases productivity}
\label{sec-5-2-1}

\begin{itemize}
\item org mode means not having to remember \LaTeX{} commands.
\item it is based on ascii text which is inherently portable.
\item Emacs!
\end{itemize}
\end{block}
\begin{block}{Why?}
\label{sec-5-2-2}

\begin{itemize}
\item Lets me focus on the Problem
\end{itemize}
\end{block}
\end{frame}
\section{columns}
\label{sec-6}
\begin{frame}
\frametitle{single column}
\label{sec-6-1}

   
\begin{columns}
\begin{column}{0.3\textwidth}
\begin{itemize}

\item Code
\label{sec-6-1-1}%
\begin{itemize}
\item one
\item two
\item three
\end{itemize}

\end{itemize} % ends low level
\end{column}
\end{columns}
\end{frame}
\begin{frame}
\frametitle{two columns}
\label{sec-6-2}
\begin{columns}
\begin{column}{0.3\textwidth}
\begin{itemize}

\item Code
\label{sec-6-2-1}%
\begin{itemize}
\item one
\item two
\item three
\end{itemize}


\end{itemize} % ends low level
\end{column}
\begin{column}{0.3\textwidth}
\begin{itemize}

\item Code
\label{sec-6-2-2}%
\begin{itemize}
\item four
\item five
\item six
\end{itemize}

\end{itemize} % ends low level
\end{column}
\end{columns}
\end{frame}
\begin{frame}
\frametitle{two columns with block}
\label{sec-6-3}
\begin{columns}
\begin{column}{0.5\textwidth}
\begin{block}{Simple block}
\label{sec-6-3-1}

\begin{itemize}
\item one
\item two
\item three
\end{itemize}
\end{block}
\end{column}
\begin{column}{0.5\textwidth}
\begin{block}{second block}
\label{sec-6-3-2}

\begin{itemize}
\item four
\item five
\item six
\end{itemize}
\end{block}
\end{column}
\end{columns}
\end{frame}
\section{code}
\label{sec-7}
\begin{frame}[fragile]
\frametitle{handling code with bable}
\label{sec-7-1}
\begin{columns}
\begin{column}{0.5\textwidth}
\begin{itemize}

\item for loop\\
\label{sec-7-1-1}%
\begin{verbatim}
#!/bin/bash
for file in $(ls)
do
    echo $file
done
exit 0
\end{verbatim}

\end{itemize} % ends low level
\end{column}
\begin{column}{0.5\textwidth}
\begin{example}[code in a block]
\label{sec-7-1-2}


\begin{verbatim}
#!/bin/bash
for file in $(ls)
do
    echo $file
done
exit 0
\end{verbatim}
\end{example}
\end{column}
\end{columns}
\end{frame}
\begin{frame}[fragile]
\frametitle{listing - c}
\label{sec-7-2}

\begin{lstlisting}[language=c]
/* a comment */
for (int i = 1; i != 10; ++i)
    std::cout << i << ": hello, world!"
              << std::endl;
\end{lstlisting}
\end{frame}
\begin{frame}[fragile]
\frametitle{listing - bash}
\label{sec-7-3}

\begin{lstlisting}[language=bash]
#!/bin/bash
# a comment
for file in $(ls)
do
    echo $file
done
exit 0
\end{lstlisting}
\end{frame}
\section{reference}
\label{sec-8}
\begin{frame}
\frametitle{all ref goes here}
\label{sec-8-1}
\end{frame}

\end{document}
