% Created 2013-12-15 Sun 10:07
\documentclass[11pt]{article}
\usepackage[utf8]{inputenc}
\usepackage[T1]{fontenc}
\usepackage{fixltx2e}
\usepackage{graphicx}
\usepackage{longtable}
\usepackage{float}
\usepackage{wrapfig}
\usepackage{soul}
\usepackage{textcomp}
\usepackage{marvosym}
\usepackage{wasysym}
\usepackage{latexsym}
\usepackage{amssymb}
\usepackage{hyperref}
\tolerance=1000
\usepackage{minted}
\makeindex
\providecommand{\alert}[1]{\textbf{#1}}

\title{Python + Emacs in Scientific computing}
\author{Scipy 2013}
\date{2013-12-14 Sat}
\hypersetup{
  pdfkeywords={scipy 2013},
  pdfsubject={Scipy 2013 notes.},
  pdfcreator={Emacs Org-mode version 7.9.3f}}

\begin{document}

\maketitle



\section{Mathematical operations in Python}
\label{sec-1}
\subsection{Square root}
\label{sec-1-1}



\begin{minted}[]{python}
1:  # ---------------------
2:  import numpy as np
3:  print np.sqrt(2)
4:  # ---------------------
\end{minted}

     1.41421356237
\subsection{Logarithm}
\label{sec-1-2}



\begin{minted}[]{python}
1:  # ---------------------------------
2:  import numpy as np
3:  print np.log(10)
4:  print np.log10(10)  # base10
5:  # ---------------------------------
\end{minted}

     2.30258509299
     1.0
\subsection{Plots}
\label{sec-1-3}


\begin{itemize}
\item Example-1
    

\begin{minted}[]{python}
1:  # ------------------------------------------------------------
2:  import matplotlib.pyplot as plt
3:  plt.plot([1,2,3,40])
4:  plt.ylabel('Y axis')
5:  plt.xlabel('X axis')
6:  plt.savefig(fname)
7:  
8:  return fname                    # return filename to org-mode
9:  # ------------------------------------------------------------
\end{minted}

     \includegraphics[width=.9\linewidth]{./plotdemo.png}
\end{itemize}


\begin{itemize}
\item Example-2


\begin{minted}[]{python}
 1:  # ------------------------------------------------------------
 2:  import matplotlib, numpy
 3:  matplotlib.use('Agg')
 4:  import matplotlib.pyplot as plt
 5:  fig=plt.figure(figsize=(4,2))
 6:  x=numpy.linspace(-15,15)
 7:  plt.plot(numpy.sin(x)/x)
 8:  fig.tight_layout()
 9:  plt.savefig('images/python-matplot-fig.png')
10:  return 'images/python-matplot-fig.png' # return filename to org-mode
11:  # ------------------------------------------------------------
\end{minted}

     \includegraphics[width=.9\linewidth]{./images/python-matplot-fig.png}
\end{itemize}
\section{Data}
\label{sec-2}
\subsection{Table: Student marks}
\label{sec-2-1}



\begin{center}
\begin{tabular}{lrrr}
 Student   &  Maths  &  Physics  &  Mean  \\
\hline
 Bertrand  &     13  &       09  &        \\
 Henri     &     15  &       14  &        \\
 Arnold    &     17  &       13  &        \\
 Sam       &     15  &       12  &        \\
 Emmy      &     20  &       11  &        \\
 Roy       &     23  &       15  &        \\
 Victor    &     11  &       15  &        \\
 Robert    &     12  &       17  &        \\
 Harper    &     16  &       10  &        \\
\hline
 Mean      &         &           &     0  \\
\end{tabular}
\end{center}





\begin{minted}[]{python}
 1:  # ------------------------------------------------------------
 2:  maths=[]
 3:  physics=[]
 4:  mean=[]
 5:  for i in marks[1:-1]:
 6:      maths.append(i[1])
 7:      physics.append(i[2])
 8:      mean.append(i[3])
 9:  
10:  import matplotlib.pyplot as plot
11:  plot.plot(physics,maths)
12:  plot.ylabel('Physics')
13:  plot.xlabel('Maths')
14:  plot.savefig('marks.png')
15:  
16:  return 'marks.png'
17:  # ------------------------------------------------------------
\end{minted}


\includegraphics[width=.9\linewidth]{./marks.png}
\subsection{Table: VI characteristics of diode marks}
\label{sec-2-2}


\begin{center}
\begin{tabular}{rrl}
 V(volts)  &  I(mA)  &  V/I  \\
\hline
     0.21  &   0.21  &  .    \\
     0.41  &   0.41  &  .    \\
     0.61  &   0.61  &  .    \\
     0.81  &   0.81  &  .    \\
     1.09  &   1.09  &  .    \\
     1.20  &   1.20  &  .    \\
\hline
\end{tabular}
\end{center}





\begin{minted}[]{python}
 1:  # ------------------------------------------------------------
 2:  v=[]
 3:  i=[]
 4:  for reading in readings[1:]:
 5:      v.append(reading[1])
 6:      i.append(reading[2])
 7:  
 8:  import matplotlib.pyplot as plt
 9:  plt.plot(i,v)
10:  plt.ylabel('I')
11:  plt.xlabel('V')
12:  plt.savefig('iv.png')
13:  
14:  return 'iv.png'
15:  # ------------------------------------------------------------
\end{minted}

\includegraphics[width=.9\linewidth]{./iv.png}

\end{document}
