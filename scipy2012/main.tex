\documentclass{beamer}
%
% Choose how your presentation looks.
%
% For more themes, color themes and font themes, see:
% http://deic.uab.es/~iblanes/beamer_gallery/index_by_theme.html
%
%% themes .sty files download
%% http://www.ctan.org/tex-archive/macros/latex/contrib/beamer/base/themes/theme
%% 
\mode<presentation>
{
  \usetheme{Frankfurt}      % or try Darmstadt, Madrid, Warsaw, ...
  %% \usecolortheme{default} % or try albatross, beaver, crane, ...
  \usecolortheme[RGB={0,104,139}]{structure}%deepskyblue
  \usefonttheme{serif}  % or try serif, structurebold, ...
  \setbeamertemplate{navigation symbols}{}
  \setbeamertemplate{caption}[numbered]
  \useinnertheme{rounded}
  %% \useoutertheme{shadow}
  %% \useoutertheme{infolines}
} 

\usepackage[english]{babel}
\usepackage[utf8x]{inputenc}
\logo{\includegraphics[height=1.2cm]{scipyshiny_small.png}}
\setbeamercovered{transparent}

% On writeLaTeX, these lines give you sharper preview images.
% You might want to comment them out before you export, though.
\usepackage{pgfpages}
\pgfpagesuselayout{resize to}[%
  physical paper width=8in, physical paper height=6in]

\title[Python on Aakash]{Python on Aakash}
\author{www.github.com/androportal}
\institute{Indian Institute of Technology, Bombay}
\date{December 29,2012}

\begin{document}

\begin{frame}
  \titlepage
\end{frame}

% Uncomment these lines for an automatically generated outline.
\begin{frame}{Outline}
 \tableofcontents
\end{frame}

\section{Introduction}
\subsection{About Aakash}
\begin{frame}{Introduction}
\begin{itemize}
  \item {\tt Aakash} - a low cost access device
  \item Specially for student to help them learn
\end{itemize}
\vskip 1cm
\begin{block}{motive}
  Should provide more than any portable device. Students can explore
  it and add their own features 
\end{block}
\end{frame}

\subsection{demo}
\begin{frame}{Demo}
\begin{itemize}
  \item aakash pic
  \item demo - \{android version\}
\end{itemize}
\end{frame}

\section{Hardware interfacing}

\subsection{GNU/linux on Aakash}
\begin{frame}{Linux}
  \begin{block}{}
    \begin{itemize}
    \item {\tt GNU/linux} and alternate OS for students and teachers
      on Aakash
    \item runs from {\tt SDcard}
    \end{itemize}
  \end{block}
  \vskip 1cm
  \centerline {\bf demo}
\end{frame}

\subsection{Arduino}
\begin{frame}{Arduino}
  \begin{block}{}
    {\tt arduino} an open-source electronics
    prototyping platform
    \end{block}
  \vskip 1cm
  \centerline {\bf demo}
\end{frame}

\subsection{expEyes}
\begin{frame}{expEyes}
  \begin{block}{}
    {\tt expEyes} A tool for learning science by exploring and experimenting
  \end{block}
  \vskip 1cm
  \centerline {\bf demo}
\end{frame}

\section{Applications on Aakash}
  \subsection{APL}
  \begin{frame}{Aakash Programming Lab}
    \begin{itemize}
      \item easy to use programming enviroment
      \item can code in {\tt Python}, {\tt C}, {\tt C++}, and {\tt Scilab}
    \end{itemize}
  \end{frame}

  \subsection{demo of APL}
  \begin{frame}{Demo}
    \begin{block}{}
      \center {\bf demo}
    \end{block}
  \end{frame}

  \subsection{Accouting on Aakash}
  \begin{frame}{Accouting on Aakash}
    \begin{itemize}
      \item An easy to use and portable Accouting software for the
        first time on Aakash
      \item robust {\tt Python} backend
    \end{itemize}
  \end{frame}

  \subsection{demo of Accouting on Aakash}
  \begin{frame}{Demo}
    \begin{block}{}
      \center {\bf demo}
    \end{block}
  \end{frame}

  


%% \section{Some \LaTeX{} Examples}
%% \subsection{Tables and Figures}
%% \begin{frame}{Tables and Figures}
%%   \begin{itemize}
%%   \item Use \texttt{tabular} for basic tables --- see Table~\ref{tab:widgets}, for example.
%%   \item You can upload a figure (JPEG, PNG or PDF) using the files menu. 
%%   \item To include it in your document, use the \texttt{includegraphics} command (see the comment below in the source code).
%%   \end{itemize}
%% \end{frame}

% Commands to include a figure:
%\begin{figure}
%\includegraphics[width=\textwidth]{your-figure's-file-name}
%\caption{\label{fig:your-figure}Caption goes here.}
%\end{figure}

%% \begin{table}
%% \centering
%% \begin{tabular}{l|r}
%% Item & Quantity \\\hline
%% Widgets & 42 \\
%% Gadgets & 13
%% \end{tabular}
%% \caption{\label{tab:widgets}An example table.}
%% \end{table}


%% \subsection{Mathematics}

%% \begin{frame}{Readable Mathematics}

%% Let $X_1, X_2, \ldots, X_n$ be a sequence of independent and identically distributed random variables with $\text{E}[X_i] = \mu$ and $\text{Var}[X_i] = \sigma^2 < \infty$, and let
%% $$S_n = \frac{X_1 + X_2 + \cdots + X_n}{n}
%%       = \frac{1}{n}\sum_{i}^{n} X_i$$
%% denote their mean. Then as $n$ approaches infinity, the random variables $\sqrt{n}(S_n - \mu)$ converge in distribution to a normal $\mathcal{N}(0, \sigma^2)$.

%% \end{frame}

\end{document}
