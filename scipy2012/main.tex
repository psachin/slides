\documentclass{beamer}
%
% Choose how your presentation looks.
%
% For more themes, color themes and font themes, see:
% http://deic.uab.es/~iblanes/beamer_gallery/index_by_theme.html
%
%% themes .sty files download
%% http://www.ctan.org/tex-archive/macros/latex/contrib/beamer/base/themes/theme
%% 
\mode<presentation>
{
  \usetheme{Frankfurt}      % or try Darmstadt, Madrid, Warsaw, ...
  %% \usecolortheme{default} % or try albatross, beaver, crane, ...
  \usecolortheme[RGB={0,104,139}]{structure}%deepskyblue
  \usefonttheme{serif}  % or try serif, structurebold, ...
  \setbeamertemplate{navigation symbols}{}
  \setbeamertemplate{caption}[numbered]
  \useinnertheme{rounded}
  %% \useoutertheme{shadow}
  %% \useoutertheme{infolines}
} 
\usepackage{array}
\usepackage{graphicx}
\usepackage{hyperref}
\usepackage{pxfonts}
\usepackage[english]{babel}
\usepackage[utf8x]{inputenc}
\titlegraphic{\includegraphics[width=4.7cm,height=3cm]{aakash-logo.png}}
\logo{\includegraphics[height=1.2cm]{scipyshiny_small.png}}
\setbeamercovered{transparent}

%% if enable below line, will \href{} or \url will not work
% On writeLaTeX, these lines give you sharper preview images.
% You might want to comment them out before you export, though.
%% \usepackage{pgfpages}
%% \pgfpagesuselayout{resize to}[%
%%   physical paper width=8in, physical paper height=6in]

\title[Python on Aakash]{Python on Aakash}
\author{Srikant \& Sachin}
\institute{Indian Institute of Technology, Bombay}
\date{December 28, 2012}

\begin{document}

\begin{frame}
  \titlepage
\end{frame}

% Uncomment these lines for an automatically generated outline.
\begin{frame}{Outline}
 \tableofcontents
\end{frame}

\section{Introduction}
\subsection{About Aakash}
\begin{frame}{Introduction}
\begin{itemize}
  \item {\tt Aakash} - a low cost access device
  \item Available for student at less than {\bf \$22}
\end{itemize}
\begin{columns}
\begin{column}{.48\textwidth}
%\color{red}\rule{\linewidth}{2pt}
\begin{block}{Aakash-2 specifications}
\begin{itemize}
  \item 1 Ghz Processor
  \item 512 MB RAM
  \item 4 GB ROM
  \item 3000 mAh battery
  \item Multi-touch, Capacitive
  \item Android ICS (4.0.4) 
  \item Wifi connectivity
  \item Front camera
  \item Accelerometer
\end{itemize}
\end{block}
\end{column}%
\begin{column}{.20\textwidth}
 \begin{block}{}
  \centerline{\href{file:///home/sachin/Videos/fossin/final.AVI}{Demo}}
 \end{block}
\end{column}%
\end{columns}
\end{frame}

\section{Applications on Aakash(Android)}
  \subsection{IPython notebook}
  \begin{frame}{IPython notebook}
    \begin{itemize}
    \item Ipython notebook with {\tt Matplotlib} works on android
      with firefox or chrome web-browser 
      \pause
    \end{itemize}
    \begin{block}{}
      \centerline{\href{file:///home/sachin/Videos/fossin/final.AVI}{Demo}}
    \end{block}
  \end{frame}
  
  \subsection{Aakash Programming Lab}
  \begin{frame}{Aakash Programming Lab}
    \begin{itemize}
      \item Full fledged programming environment for {\tt
        Python-2.7.2} and other languages
      \item {\tt Scilab-5.4}, an Open source software for numerical
        computation
      \item Was extended to {\tt C} and {\tt C++} with {\tt GCC-4.6.3}
      \item Offline docs and reference manual 
        \pause
    \end{itemize}
    \begin{block}{}
      \centerline{\href{file:///home/sachin/Videos/fossin/final.AVI}{Demo}}
    \end{block}
  \end{frame}

  \subsection{Accounting on Aakash}
  \begin{frame}{Accounting on Aakash}
    \begin{itemize}
      \item Derived from {\tt GNUkhata}
      \item Portable accounting platform on Android intended for
        Accountants and Students.
      \item Easy to use UI
        \pause
        \begin{itemize}
          \item {\tt Android-xmlrcp} client side ({\tt java)})library
          \item {\tt pysqlite2} binding
        \end{itemize}
        \pause
    \end{itemize}
    \begin{block}{}
      \centerline{\href{file:///home/sachin/Videos/fossin/final.AVI}{Demo}}
    \end{block}
  \end{frame}

  \subsection{Challenges}
  \begin{frame}{Challenges in Android}
    \begin{block}{Aakash Programming lab}
      \begin{itemize}
        \item CPU and memory optimization
        \item Unable to compile {\tt scilab} using Android NDK
        \item Plots in {\tt Scilab}
        \item Integrating {\tt shellinabox} server
      \end{itemize}
    \end{block}
    \end{frame}

%  \subsection{Challenges contd..}
  \begin{frame}{Challenges in Android}
    \begin{block}{IPython}
      \begin{itemize}
        \item Unavailability of web-sockets in default Android webkit
          browser
        \item IPython-kernel random shutdown
      \end{itemize}
    \end{block}
    \begin{block}{Accounting software on Aakash}
      \begin{itemize}
        \item {\tt sqlite} - as a database in GNUkhata
        \item client({\tt java})-server({\tt xmlrpc}) interaction
      \end{itemize}
    \end{block}
  \end{frame}

\section{Applications on Aakash(linux)}
%\subsection{linux on Aakash}
\begin{frame}{Linux}
  \begin{block}{}
    \begin{itemize}
    \item {\tt linux} and alternate OS for students and teachers on
      Aakash
    \item runs entirely from {\tt SDcard}
      \pause
    \end{itemize}
  \end{block}
%  \vskip 1cm
  \begin{block}{}
    \centerline{\href{file:///home/sachin/Videos/fossin/final.AVI}{Demo}}
  \end{block}
\end{frame}

\subsection{Arduino}
\begin{frame}{Arduino on Aakash}
  \begin{block}{Arduino}
    An open-source electronics prototyping platform
  \end{block}
%  \vskip 1cm
  \begin{block}{}
    \centerline{\href{file:///home/sachin/Videos/fossin/final.AVI}{Demo}}
  \end{block}
\end{frame}

\subsection{expEyes}
\begin{frame}{expEyes on Aakash}
  \begin{block}{expEyes}
    A tool for learning science by exploring and experimenting
  \end{block}
%  \vskip 1cm
 \begin{block}{}
    \centerline{\href{file:///home/sachin/Videos/fossin/final.AVI}{Demo}}
  \end{block}
\end{frame}

 \subsection{Challenges}
  \begin{frame}{Challenges in linux}
    \begin{block}{}
      \begin{itemize}
        \item touch screen drivers
        \item USB-to-serial, cdc drivers
        \item Desktop environment
        \item GPU drivers
        \item kernel level hacks, module loading, wifi etc.
        \item file-system
%        \item {\tt script.bin}
      \end{itemize}
    \end{block}
    \end{frame}
  
  \section{links and downloads}
  \subsection{Android apps on Aakash}
  \begin{frame}{Android apps on Aakash}
    \begin{block}{\url{https://github.com/androportal}}
      \begin{itemize}
      \item IPython notebook \hfill \url{/apk-ipython}
      \item Aakash Programming Lab \hfill \url{/APL-apk}
        \begin{itemize}
        \item docs \hfill \url{/docs}
        \end{itemize}
      \item Accounting on Aakash \hfill \url{/gkAakash}
      \end{itemize}
    \end{block}
  \end{frame}

  \subsection{Linux on Aakash}
  \begin{frame}{linux on Aakash}
    \begin{block}{}
      \centerline
          {\url{http://androportal.github.com/linux-on-aakash}}
    \end{block}
  \end{frame}

%  \subsection{Questions}
  \begin{frame}{}
      \centerline{\bf Questions?}
  \end{frame}
  

%% \section{Some \LaTeX{} Examples}
%% \subsection{Tables and Figures}
%% \begin{frame}{Tables and Figures}
%%   \begin{itemize}
%%   \item Use \texttt{tabular} for basic tables --- see Table~\ref{tab:widgets}, for example.
%%   \item You can upload a figure (JPEG, PNG or PDF) using the files menu. 
%%   \item To include it in your document, use the \texttt{includegraphics} command (see the comment below in the source code).
%%   \end{itemize}
%% \end{frame}

% Commands to include a figure:
%\begin{figure}
%\includegraphics[width=\textwidth]{your-figure's-file-name}
%\caption{\label{fig:your-figure}Caption goes here.}
%\end{figure}

%% \begin{table}
%% \centering
%% \begin{tabular}{l|r}
%% Item & Quantity \\\hline
%% Widgets & 42 \\
%% Gadgets & 13
%% \end{tabular}
%% \caption{\label{tab:widgets}An example table.}
%% \end{table}


%% \subsection{Mathematics}

%% \begin{frame}{Readable Mathematics}

%% Let $X_1, X_2, \ldots, X_n$ be a sequence of independent and identically distributed random variables with $\text{E}[X_i] = \mu$ and $\text{Var}[X_i] = \sigma^2 < \infty$, and let
%% $$S_n = \frac{X_1 + X_2 + \cdots + X_n}{n}
%%       = \frac{1}{n}\sum_{i}^{n} X_i$$
%% denote their mean. Then as $n$ approaches infinity, the random variables $\sqrt{n}(S_n - \mu)$ converge in distribution to a normal $\mathcal{N}(0, \sigma^2)$.

%% \end{frame}

\end{document}
