%% workshop @t icfoss, 16 and 17th Feb 2013
\documentclass{beamer}
%% 
% Choose how your presentation looks.
%
% For more themes, color themes and font themes, see:
% http://deic.uab.es/~iblanes/beamer_gallery/index_by_theme.html
%
%% themes .sty files download
%% http://www.ctan.org/tex-archive/macros/latex/contrib/beamer/base/themes/theme
%% 
\mode<presentation>
{
  \usetheme{Frankfurt}      % Frankfurt, Darmstadt, Madrid, Warsaw
  \usecolortheme{default} % or try albatross, beaver, crane, ...
  %% \setbeamercolor{}{fg=, bg=black}
  %% \setbeamercolor{frametitle}{fg=default, bg=black}
  \setbeamercolor{upper separation line head}{bg=r}

  \usecolortheme[RGB={0,104,139}]{structure}%deepskyblue
%  \usecolortheme[RGB={59,89,152}]{structure}%facebook
  \usefonttheme{serif}  % or try serif, structurebold, ...
  \setbeamertemplate{navigation symbols}{}
  \setbeamertemplate{caption}[numbered]
  \useinnertheme{rounded}
}
%% import packages
\usepackage{array}
\usepackage{graphicx}
\usepackage{hyperref}
\usepackage{pxfonts}
\usepackage[english]{babel}
\usepackage[utf8x]{inputenc}

\usepackage{amsmath}
 %\usepackage{SIunits}
 \usepackage[sc]{mathpazo}
 \usepackage{tikz}
 \usetikzlibrary{arrows,shapes}

 \titlegraphic{\includegraphics[width=4.7cm,height=3cm]{aakash-logo.png}}
 \logo{\href{http://icfoss.in/}{\includegraphics[height=0.5cm, width=1cm]{icfoss_logo.jpg}}}
 \setbeamercovered{transparent}
 %% 
 \title[Aakash application development]{Aakash application development}
 \author{\url{http://aakashlabs.org}}
 \institute{Indian Institute of Technology, Bombay}
 \date{February 17, 2013}
 %% 
 %% 
 \begin{document}
 %% 
 \begin{frame}
   \titlepage
 \end{frame}
 %% 
 %% Uncomment these lines for an automatically generated outline.
 \begin{frame}{Outline}
  \tableofcontents
 \end{frame}
 %% 
 %% 
 \section{GNU/linux on Aakash}
 \begin{frame}{GNU/linux as an alternate OS on Aakash}
   \begin{block}{}
     \begin{itemize}
       \item runs entirely from {\tt SDcard} in native mode
       \item not all FOSS applications can be ported {\em as it is}
       \item Android is not the only OS on phones/tablets \pause
         \begin{itemize}
         \item Ubuntu on phones
         \item Firefox OS
         \end{itemize}
     \end{itemize}
   \end{block}
 \end{frame}

 \begin{frame}{Preparing your GNU/linux build}
   %% cc
   %% uboot
   %% kernel
   %% fs
   \begin{block}{cross compiler toolchain}
     \begin{itemize}
     \item download cross-compiler toolchain
     \item {\tt sudo dpkg-reconfigure -plow dash}
     \item {\tt export PATH=</your/codesourcery/bin/path>:\$PATH }
     \end{itemize}
   \end{block}

   \begin{block}{uboot}
     \begin{itemize}
     \item {\tt git clone -b sunxi https://github.com/androportal/uboot-allwinner.git}
     \item {\tt sudo apt-get install u-boot-tools}
     \item {\tt make a13\_olinuxino CROSS\_COMPILE=arm-none-linux-gnueabi-}
     \end{itemize}
   \end{block}
 \end{frame}

 
 \begin{frame}{Preparing your GNU/linux build}
   \begin{block}{kernel dependencies}
     \begin{itemize}
     \item {\tt fakeroot, build-essential, gcc, kernel-package, make,
       automake, libncurses5-dev, git}
     \end{itemize} 
   \end{block}

   \begin{block}{compilation}
     \begin{itemize}
     \item {\tt make ARCH=arm a13\_defconfig}
     \item {\tt make ARCH=arm menuconfig}
     \item {\tt make ARCH=arm CROSS\_COMPILE=arm-none-linux-gnueabi-
       uImage}
     \item {\tt make ARCH=arm CROSS\_COMPILE=arm-none-linux-gnueabi-
       INSTALL\_MOD\_PATH=out modules}
     \item {\tt make ARCH=arm CROSS\_COMPILE=arm-none-linux-gnueabi-
       INSTALL\_MOD\_PATH=out modules\_install}
     \end{itemize} 
   \end{block}
 \end{frame}

 \begin{frame}{Preparing your GNU/linux build}
   \begin{block}{root filesystem}
     \begin{itemize}
     \item {\url
       http://cdimage.ubuntu.com/ubuntu-core/releases/12.10/release/ubuntu-core-12.10-core-armhf.tar.gz}
     \item {\tt qemu-user-static}
       %% ch-mount.sh script
     \end{itemize}
   \end{block} 

   \begin{block}{Preparing your SDcard}
     \begin{itemize}
     \item partioning
     \item copy uboot
     \item copy kernel
     \item releasing your GNU/linux image
     \end{itemize}
   \end{block}
 \end{frame}
  
 %% \subsection{GNU/linux application}
 %% \begin{frame}{GNU/linux application development}
 %%   \begin{block}{}
 %%     \begin{itemize}
 %%     \item no point
 %%     \end{itemize}
 %%   \end{block}
 %% \end{frame}

 %% \subsection{debug GNU/linux application}
 %% \begin{frame}{GNU/linux application debugging}
 %%   \begin{block}{}
 %%     \begin{itemize}
 %%     \item debug GNU/linux app
 %%     \end{itemize}
 %%   \end{block}
 %% \end{frame}

 \section{Version control}
 \begin{frame}{Managing your application using version control}
   \begin{itemize}
   \item why version control ??
   \item types of version control 
   \end{itemize}

   \begin{block}{{\tt git}}
     \begin{itemize}
     \item entirely written in {\tt C}
     \item recommended for large projects, eg. {\tt linux kernel}
     \item the way {\tt git} handles {\tt branches}, and {\tt
       conflicts} is {\em owesome!!}
     \item {\url{github.com}}, {\url{gitorious.org}}, etc
     \end{itemize}
   \end{block}
 \end{frame}

 \subsection{hands-on {\tt git}}
 \begin{frame}{using {\tt git}}
     \begin{itemize} \pause
     \item initialize you project \pause
     \item adding files to {\tt git} \pause
     \item commit your changes \pause
     \item pushing it to remote host \pause
     \item revert to know state \pause
     \item branching and merge \pause
     \item handling conflicts \pause
     \item pull requests \pause
     \item clone 
     \end{itemize}
     %    share .gitconfig, .gitignore_global and .gitmessage.txt file
 \end{frame}

 \section{Documentation}
 \subsection{why do we need Documentation ??}
 \begin{frame}{why document our project?}
     \begin{itemize}
     \item No documentation, No project!
     \item User and developer's guide
     \item Contributing to existing code 
     \end{itemize}
 \end{frame}

 \subsection{Documentation using {\tt sphinx}}
 \begin{frame}{{\tt sphinx}}
   \begin{itemize}
   \item plain text({\tt .rst})
   \item even {\bf Documentation} can be {\em version controlled}
   \end{itemize}
   \begin{block}{}
     \begin{itemize}
     \item {\tt sudo apt-get install python-sphinx}
     \item {\tt sphinx-quickstart} \pause
     \item {\tt sudo apt-get install rst2pdf}
     \item {\tt pandoc}
     \end{itemize} \pause
   \end{block}
   \begin{block}{}
     \centerline{{\tt {\url http://docs.python.org/devguide/documenting.html}}}
   \end{block}
 \end{frame}

\section{App release mechanism}
\subsection{how to release apps}
\begin{frame}{app release}
    \begin{itemize}
    \item {\tt play.google.com}
    \item host it on {\tt github.com} or your custom page
    \item releasing version ({\tt git tag})
    \end{itemize}
\end{frame}

%% \subsection{Questions}
\begin{frame}{}
  \centerline{\bf Questions?}
\end{frame}

\begin{frame}{References and links}
    \begin{itemize}
    \item {\url github.com/androportal/linux-on-aakash}
    \item {\url http://git-scm.com/book}
    \end{itemize}
\end{frame}
\end{document}

